%!TEX root = ../thesis_a4.tex

\chapter{Introduction}
\label{chap:intro}

\blockcquote[]{Varese1966}{``\textit{What is music but organised noises?}''}

If music is indeed, as Varèse suggested, to be distilled summarily to “organised noises” in time, then surely the art of music composition has amongst its most effective tenets the dialectical balance between repetition and variation. For a musical motif, idea or theme to be memorable, the composer must use repetition to exploit the human's application of memory in establishing patterns and context between sequences of notes, or so maintains Levitin:

\blockcquote[]{Levitin}{``\textit{Repetition, when done skillfully by a master composer, is emotionally satisfying to our brains, and makes the listening experience as pleasurable as it is.}''}

But to dispel boredom and make it really interesting and engaging it unquestionably needs to be varied; the composer must carefully choose which musical parameters are coaxed from their centres to confound expectation and break new ground within the oeuvre. Even as the unabashed antagoniser of tradition, Pierre Boulez, concedes:

\blockcquote[]{Campbell1997}{``\textit{...a high level of interest in repetition and variation (analogy and difference, recognition and the unknown) is characteristic of all musicians.}''} 

For common practice period Western Art music, repetition is used to turn melodic and rhythmic fragments into concrete themes, but repetition of course is also dominantly utilised in macrostructural arrangement of form, most notably in the ternary-based sonata - where thematic material is introduced, developed and recapitulated in a systematic manner that aids listeners navigate dense and complex works \citep{Benward2008}.

Later, African-American roots music such as blues and R\&B would centre its aesthetic on bite-sized, guitar-driven ostinatos known as "riffs", subsequently sowing the seeds for the explosion of rock and pop culture \citep{Middleton2009, Hatch1987}. Minimalist composers such as Steve Reich, Philip Glass, LaMonte Young and Terry Reilly \citep{Reich2011, Nyman1999} would push the mantra of "doing a lot with a little" through exaggerated repetition to its extremes\footnote{Reich, like his predecessor Varèse gained notoriety for instigating "the last great musical scandal of the 20th century" \citep{ross2007rest}. During the concert premiere of his seminal Four Organs an audience member was purported to have ran down the aisles, screaming “All right, I confess!”.}, mirroring prior movements in literature (e.g. Beckett) and the visual arts. Minimalism, in its essence, tips the weight of repetition and variation; repetition is so deeply ingrained and rife in the process\footnote{Nyman and Reich both use the term ``process music'' in distinguishing their work as well as that of Cage from works by post-war serialists and perhaps more maximalists such as Xenakis and Stockhausen} that any slight change or variation introduced is overwhelmingly noticeable and immensely consequential.

The stylistic intentions of this thesis however, are largely concerned with Electronic Dance Music - a subset of electronic music first appearing roughly in the 1980s intended initially for dancing at events like raves or nightclubs \citep{McLeod2001, Butler2006}. In dance music\footnote{Nomenclature will be dealt with in the following chapter} small, repeated patterns known as "loops" provide the core building blocks for compositions
	\citep{Neill2002, Patricio2016}. A single idea, perhaps consisting of a drum pattern, a keyboard vamp or often a recontextualised, “borrowed” audio sample - as a result of careful listening to myriad musical or sonic sources - can become the seed for an entire track\footnote{We shall see - in the case of drum and bass - how a particular motif can also spawn an entire subgenre within dance music.}, as Neill observes:
	
\blockcquote[]{Neill2002}{``\textit{Just as composers in earlier historical periods often worked within a given set of large-scale formal parameters (sonata form, dance forms, tone poems etc.), innovative pop electronic composers use steady pulse, loop-based structures and 4/4 time as a vehicle for a wide range of compositional ideas and innovations}''}	
	
Electronic dance music is ultimately a product of the technology on which it is created during a nascent time span, a music born out of the limitations of the tools available. Initially these tools comprised synthesisers, samplers and drum machines, but as the digital computer became viable for consumers, they increasingly involved sophisticated software that virtualises modern recording studio capability. 

So with a music so seemingly mechanical and so intrinsically coupled with the machine, naturally it is pertinent to ask: are the machines themselves capable of fulfilling the role of the composer? Or to put it in other words, rather than the composer choosing and organising the sounds that the systems generate, can we go one step further and call on the machine to choose and organise sounds also? Of course what we are describing here is algorithmic \citep{Roads1996}, or generative music \citep{Collins2008} - where computer programs are used to compose music - a topic that has fascinated researchers, composers and musicologists alike for over half a century \citep{Burns1994, Fernandez2013}, undoubtedly preceding the global rise of dance music culture. Successful automatic algorithmic composition of works are heavily dependent on the intended style and the algorithmic techniques chosen. Building on a wealth of knowledge historically available in matters of harmonic counterpoint and voice leading, \cite{Cope1991}, for example, has for some time now created convincing syntheses of Bach chorales, but acceptable attempts at modelling modern pop music are only recently beginning to emerge \citep{Ghedini2015}. Attempting to capture the entire gamut of creative musical endeavour with computational means seems a lofty, and for now, rather intractable prospect, but what about processes that offer the composer a helping hand?

Thus, in this thesis I examine computational, algorithmic or otherwise generative approaches that assist the composer in creating rhythm-centric loops suitable for dance music production. I emphasise \textit{assist} to contrast with traditional algorithmic compositional systems that may focus on producing complete works or output with little or no interaction or intervention from the user other than setting some initial parameters. As Pasquier explains when introducing what he terms “musical metacreation”:

\blockcquote[]{Pasquier2017}{``\textit{Such systems will tend to produce – ideally musically successful – variations of the same composition rather than creating completely novel outputs with each run.}''}

Like Pasquier, I emphasise assist to describe situations where the composer has an idea (i.e. a loop) they want to expand and develop through intelligent and systematic repetition and variation, while maintaining the impression that what is generated remains truly something they can call their own, rather than any deus ex machina.

My process is an iterative one, perhaps, echoing the evolution of dance music production practices, which has itself moved from primitive symbolic control of hardware devices to more signal focussed methods that manipulate and process sampled sound. I draw from knowledge in the literature, propose advancements where they are lacking, build prototypes, then most crucially, evaluate the work with other active users. 

Initially I describe how symbolic methods, coupled with perceptual knowledge of similarity between rhythm representations, can be combined to create a perceptually motivated generative drum machine that seeks to create convincing patterns. Following a user evaluation of the system, I highlight some limitations we see in purely symbolic domains. These limitations I feel justify a shift to more content-based techniques such as concatenative synthesis, in order to build systems that can harness state of the art research in machine listening and the thriving field of \acrfull{mir} for repurposing existing audio in a logical and efficient manner. We describe concatenative synthesis from a research perspective, and offer some novel improvements for addressing some shortcomings in \acrfull{hmm} driven approaches, for instance. Subsequently, a system is proposed from the user perspective, demonstrating clearly how to distil the state of the art techniques typically found in research oriented concepts, to deliver a tool that is visually and creatively appealing, compelling and viable for the modern music producer. 

For now though, the rest of the introduction will continue with a definition of my motivations and the wider context of my work within the GiantSteps project The introductory chapter concludes with a detailed overview of the contents of each subsequent chapter in the dissertation.

\section{Motivations and The GiantSteps Project}

%Most of my own artistic output I would broadly categorise as ``beatless'' or ``beat-oriented''. My ``beatless'' music tends to be electroacoustic music written for fixed media, or some combination of electronics with solo instruments or ensembles\footnote{\url{http://cmmr2017.inesctec.pt/programme/concerts/}}. Nearly all of these efforts involve some labour divested through formal computational processes, whether this be real-time compositional algorithms \citep{Nuanain2014}  or advanced analysis and resynthesis procedures afforded by musical programming languages and environments. These works are produced largely within the parameters I define, or perhaps based on some crude personal abstraction of some well-established musical form  serialism or aesthetic like minimalism. 
%
%Over the years I have also listened to to more and more electronica and dancefloor music (which I will summarise  as well as attending . My curiosity extended to its composition. Naively I assumed that with its machine-centric rigitivity and perceived primitiveness compared to more ``serious'' music I could 

The GiantSteps project is a European Union led initiative launched in 2013 with the intention of creating the ``Seven-League Boots'' for music production in the next decade and beyond \citep{Knees2016a}. Three main goals of the project embrace:

\begin{enumerate}
  \item Developing and integrating musical expert agents, supportive and inspirational systems for melody, harmony, rhythm, structure or style, based on symbolic, audio and metadata analysis.
  \item Developing improved interfaces and paradigms for musical human-computer interaction and collaborative control of multi-dimensional parameter spaces using novel visualisation techniques.
  \item Addressing low cost portable devices by developing low-complexity algorithms for music analysis and recommendation tailored to their capabilities.
\end{enumerate}

The research consortium consists of multidisciplinary bodies drawing from academia, industry and the arts, including Universitat Pompeu Fabra, Johannes Kepler University, \acrshort{steim}, Reactable Systems and Native Instruments. One of the primary successes of the project has been the constant focus on user analysis, that has been pivotal in helping guide and shape the advancement of the state of the art. Through active engagement at workshops, conferences and festivals worldwide such as Red Bull Music Academy, Sónar, Music Hack Day, considerable effort has been devoted to study DJs, producers and composers needs, desires, and skills; investigating their processes and mental representations of tasks and tools; and evaluating their responses to prototypes that consolidate research into practical realisations. 

\section{Thesis Outline}

Before delving into the science, \chapref{chap:dancemusic} offers a critical introduction to electronic dance music from a cultural and musicological perspective. Dance music is a relatively recent phenomenon compared its precursory roots in electroacoustic and contemporary composition, already the subject of many key texts as well as devoted gatherings for its academic discourse. Furthermore the context for computational and musicological analysis for earlier eras of Art music as well as popular styles like rock and jazz are historically well-established. We therefore offer the case for \textit{why} electronic dance music is interesting to study, how its various subgenre strands have developed and its compositional traits which shall hopefully inform our analysis and enventual synthesis using computational means. 

In \chapref{chap:symbolic} we will begin our journey by introducing algorithmic composition or generative music as that corner of computer music concerned with using computational resources and programmatic techniques to derive aesthetically pleasing music. As we delve further into this we reveal how natural aspects of music such as rhythm can be recreated using symbolic representational structures informed by the study of how we perceive such facets when they are performed or composed by humans. We propose a system that can recreate such rhythms automatically, embedding our interpretation of this knowledge within an interactive genetic algorithm aimed at fulfilling the needs of electronic dance music producers. We pay close attention to how one can evaluate such systems from a user perspective. Based on this user evaluation, we give a frank appraisal of our own personal impressions of the system and explain how this motivates a change of direction and focus.

\chapref{chap:sota} provides a contextual basis for our shift of focus by introducing state of the art techniques in art and in academic literature that demonstrates the aesthetic reuse of content. By content we mean existing objects that exist in the wild that can be exploited, shifted and reused, malleable for repurposing in creating new work. In terms of electronic music, we describe the particular practice of sampling - whereby existing sounds and music are directly adapted and recontextualised in the aforementioned manner - tracing its historical trajectory from the early days of tape experiments through to modern dance music and contemporary efforts. As is most pertinent to the ambitions of this thesis, we are expressly concerned with those works that seek to automate or aid the composer to perform sampling in some systematic or hopefully intelligent manner using computer resources. This leads us naturally to concentrate on granular synthesis initially, with the second portion of the chapter extensively reviewing state of the art in concatenative synthesis.

With this context in place, \chapref{chap:pyconcat} delves in depth into the practical aspects of crafting and experimenting with concatenative systems. Building on the state of the art methods introduced in \chapref{chap:sota}, theoretical, algorithmic and mathematical mechanisms are detailed. In the area of unit selection - one of core tasks in concatenative synthesis - some shortcomings are highlighted concerning the application of \acrfull{hmm}s, so some novel techniques are proposed to overcome them. The PyConcat library is delivered as a culmination of our efforts in this course of study, intended as a research focussed, easy-to-use framework for exploring concatenative synthesis’ fundamental concepts.

In \chapref{chap:rhythmcat} we return to one of the other goals of our thesis: adapting state of the art research from academia to deliver new musical software that considers the needs of our user (rather than the researcher). To this end, the RhythmCAT system is presented, an easy to use virtual software instrument that encapsulates the sampling capabilities of concatenative synthesis along with some new visual and interaction paradigms that seeks to enhance the creativity of the modern electronic dance music producer and composer. 

As purported in \chapref{chap:evaluation}, the systematic evaluation of computer music platforms is an oft-neglected activity, especially in academic efforts where the designer might also be the sole user. Since Our work in \chapref{chap:rhythmcat} is aimed squarely at other users, thus there needs to be a well thought out framework in place to validate it for its efficacy and efficiency in addressing the needs of those users. This chapter is devoted to devising such a framework, that we hope delivers a substantial contribution to the music software making community working towards similar goals.

To conclude the thesis, \chapref{chap:conclusions} reflects on the overall scope and findings of the research delivered. We summarise the achievements, highlight some criticisms and lay a clear path for continuing with future work. 


 
