
\chapter{Resum}

La pràctica de la composició musical part sovint d'una petita idea o motiu que floreix en una obra completa mitjançant una acurada aplicació de la repetició, variació i una mica d'inspiració. En música electrònica de ball (que és altament repetitiva i basada en el ritme), aquest motiu conegut com \textit{loop} és el seu component fonamental, i adquireix una importància més gran que en qualsevol altre estil. Mentre altres gèneres poden dependre de regles establertes per regir l'estructura harmònica i la forma, els temes de música electrònica de ball no s'allunyen molt del seu motiu central fonamental en un sentit tradicional. Aquesta particular música s'elabora a través de complexos arranjaments de capes tímbriques i d'intensitat que sorgeixen a partir d'una aplicació lliure de l'activitat rítmica i dels efectes de so.

Aquesta tesi explora mètodes computacionals per a la generació i variació de l'activitat rítmica, que és un peça clau en la composició de \textit{loops} efectius. Comencem en el domini simbòlic, basant-nos en múltiples mètodes clàssics de composició algorítmica, i en el coneixement més avançat sobre percepció de similitud rítmica per a construir \textit{GenDrum}, una caixa de ritmes intel·ligent basada en algoritmes genètics. Una avaluació amb oients demostra que és un enfocament vàlid, però ens qüestionem l'habilitat de mètodes purament simbòlics per a capturar l'essència del timbre.

Els mètodes més moderns de composició de música electrònica es distingeixen per un ús liberal del samplejat i l'apropiació de sons preexistents, així com del disseny de nous sons purament sintetitzats. La síntesi concatenada aplica regles d'alt nivell i criteris que intenten combinar frases de sons d'una manera més intel·ligent i sistemàtica. Aquest enfocament basat en contingut utilitza la recerca en recuperació d'informació musical per treballar directament amb àudio i amb una codificació latent i multidimensional de l’espectre i de les característiques tímbriques.

Com a segona contribució d'aquesta tesi, examinem la rellevància i aplicació de la síntesi concatenada per a producció de música electrònica de ball. Presentem una revisió exhaustiva de treballs clau en l'àrea de la síntesi concatenada i resumim els fonaments algorítmics en què se solen basar aquests sistemes. \textit{PyConcat}, un entorn per a síntesi orientat a la recerca, encapsula molts dels mètodes més importants juntament amb algunes millores per a la selecció de característiques tímbriques, i la selecció d'unitats amb Models Ocults de Markov.

Però, especialment, segueixen sent clau les implicacions en considerar les necessitats dels \ textit {usuaris} en el disseny de sistemes concatenats, i el lloc que aquests ocupen en els seus processos compositius i de producció. La contribució final de la tesi és \textit{RhythmCAT}, un instrument virtual visualment atractiu i amb una metàfora d'interacció única per a l'exploració de la síntesi concatenada d'una manera pràctica i fàcil d'utilitzar. Aquest èmfasi en les implicacions de l'usuari de la síntesi concatenada porta naturalment a la qüestió de l'avaluació, tant en el nostre sistema com en l'ampli ecosistema d'agents computacionals generatius i creatius. Proposem una altra metodologia d'avaluació que amplia la realitzada en el domini simbòlic, i discutim els resultats en detall. Finalment, presentem les conclusions sobre la nostra feina i possibles direccions per a treballs futurs.


\vfill
{\noindent (\emph{Translated from English by Juanjo Bosch})}