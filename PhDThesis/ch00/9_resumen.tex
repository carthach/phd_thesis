
\chapter{Resumen}

La práctica de la composición musical parte a menudo de una pequeña idea o motivo que florece en una obra completa mediante una cuidadosa aplicación de la repetición, variación y una pizca de inspiración. En música electrónica de baile (que es altamente repetitiva y basada en el ritmo), dicho motivo, conocido como \textit {loop} es su componente fundamental, y adquiere una importancia mayor que en cualquier otro estilo. Mientras otros géneros pueden depender de reglas establecidas para regir la estructura armónica y la forma, las piezas de música electrónica de baile no se alejan mucho de su motivo central fundamental en un sentido tradicional. Esta característica música se elabora a través de complejos arreglos de capas de timbres e intensidad que surgen a partir de una aplicación libre de la actividad rítmica y de efectos de sonido.

Esta tesis explora métodos computacionales para la generación y variación de la actividad rítmica, que es un pieza clave en la composición de \ textit{loops} efectivos. Comenzamos en el dominio simbólico, basándonos en múltiples métodos clásicos de composición algorítmica y en el conocimiento más avanzado sobre percepción de similitud rítmica, para construir \textit{GenDrum}, una caja de ritmos inteligente basada en algoritmos genéticos. Una evaluación con oyentes demuestra que es un enfoque válido, pero nos cuestionamos la habilidad de métodos puramente simbólicos para capturar la esencia del timbre.

Los métodos más modernos de composición de música electrónica se distinguen por un uso liberal del sampleado y de la apropiación de sonidos preexistentes, así como del diseño de nuevos sonidos puramente sintetizados. La síntesis concatenativa aplica reglas de alto nivel y criterios que intentan combinar frases de sonidos de una manera más inteligente y sistemática. Este enfoque basado en contenido utiliza la investigación en recuperación de información musical para trabajar directamente con audio y con una codificación latente y multidimensional del espectro y de las características tímbricos.

Como segunda contribución de esta tesis, examinamos la relevancia y aplicación de la síntesis concatenativa para producción de música electrónica de baile. Presentamos una revisión exhaustiva de trabajos clave en el área de la síntesis concatenativa y resumimos los fundamentos algorítmicos en los que se suelen basar estos sistemas. \ textit {PyConcat}, un entorno para síntesis orientado a la investigación, encapsula muchos de los métodos más importantes junto con algunas mejoras para la selección de características tímbricas, y para la selección de unidades con Modelos Ocultos de Markov .

Pero, especialmente, siguen siendo clave las implicaciones al considerar las necesidades de los \ textit {usuarios} en el diseño de sistemas concatenativos y el lugar que estos ocupan en sus procesos compositivos y de producción. La contribución final de la tesis es \ textit {RhythmCAT}, un instrumento virtual visualmente atractivo y con una metáfora de interacción única para la exploración de la síntesis concatenativa de una manera práctica y fácil de usar. Este énfasis en las implicaciones del usuario de la síntesis concatenativa lleva naturalmente a la cuestión de la evaluación, tanto en nuestro sistema como en el amplio ecosistema de agentes computacionales generativos y creativos. Proponemos otra metodología de evaluación que amplía la realizada en el dominio simbólico, y discutimos los resultados en detalle. Finalmente, presentamos las conclusiones sobre nuestro trabajo y posibles direcciones para trabajos futuros.

\vfill
{\small \noindent (\emph{Translated from English by Juanjo Bosch})}