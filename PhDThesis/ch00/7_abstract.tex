
\chapter{Abstract}

The practice of music composition often stems from a small idea or motif that blossoms into a complete work through careful application of repetition, variation and a dash of inspiration. In the highly rhythmic and repetitive strain of music that is electronic dance, this motif is the fundamental building block known as the \textit{loop}, and is accorded greater importance than any other style. While other genres can depend upon long-established rules governing things like harmonic structure and form, many dance tracks do not stray far from its fundamental central theme in the traditional sense. Rather, the craft of this unique music is revealed through complex, layering arrangements of timbre and intensity arising from the liberal application of rhythmic activity and sonic effects.

This dissertation explores computational methods for generating and varying that rhythmic activity that is so pivotal in composing effective loops. We begin the journey in the symbolic domain, and draw upon a wealth of historical methods for algorithmic composition coupled with the state of the art in rhythmic similarity perception to build \textit{GenDrum}, an intelligent drum machine using genetic algorithms. Our listener survey reveals the validity of the approach, but we question the general adroitness of purely symbolic means in capturing the acute essence of timbre.

Modern approaches to composing electronic music are distinguished by the liberal use of sampling and appropriation of existing sounds as well as the design of purely synthesised new sounds. Concatenative synthesis applies high-level rules and criteria that seek to combine phrases of sounds together in a more intelligent and systematic manner. It is a content-based approach that uses music information retrieval research to work directly with audio and its latent encoding of multidimensional spectral and timbral character.

In the second contribution of the thesis we examine the relevance and application of concatenative synthesis in the specific context of electronic dance music production. We present a comprehensive review of key works in the area of concatenative synthesis and summarise the algorithmic underpinnings that tend to drive these systems. \textit{PyConcat}, a research oriented synthesis framework, is offered to encapsulate many of the key approaches along with some novel improvements to the field and state of the art in the area of timbral feature choice as well as unit selection with Hidden Markov Models.

But, above all, the concern remains with the implications of designing concatenative systems that consider the needs of the \textit{user} and the place of such systems in their existing composition and production workflows. The final contribution of the thesis delivers \textit{RhythmCAT}, a visually appealing virtual instrument plugin with a unique interaction metaphor for exploring concatenative synthesis in a practical, user-friendly manner. This focus on the user implications of concatenative synthesis leads naturally to the question of evaluation: not only in our system but also the wider ecosystem of generative and creative computational agents. Another evaluation methodology is proposed that expands on the one conducted in the symbolic domain and the results are presented and discussed in detail. The thesis concludes with our critical impressions of the work presented and the possible directions for future study.

